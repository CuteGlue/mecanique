%% ceci est un commentaire 
%% il faut toujours commencer par \documentclass[type de papier, taille de texte]{sytle du document}
\documentclass[a4paper,10pt]{report} %%%% sytle du document : report/book/article


%%%%%%%%%%%%%%%%%%%%%%%%%%%%%%%%%%%%%%%%%%%%%%%%%%%%%%%%%%%%%%%%%%%%%%%%%%%%%%%
%% la suite est une collection des "package" ou des "librararies" pour utiliser des codes spécifiques

%% il vous suffit de les copie-coller quand vous créez un nouveau document TeX (ou bien en ajouter plus si besoin).
\usepackage[utf8]{inputenc} %% pour les accents en français
\usepackage[frenchb]{babel} %% pour un format français
\usepackage{graphicx} %% pour afficher des graphiques
\usepackage{amsmath} %% pour écrire des symboles (maths), des équations, etc.
\usepackage{amssymb}
\usepackage{color}
\usepackage{bm} %% pour lister des citations/la biblio
\usepackage{hyperref} %% pour inserer des liens internet
\usepackage{cleveref} %% pour faire des références uax équations, tableaux, etc.
%\usepackage{setspace} %% pour changer l'espace entre les lignes
%\linespread{1.6} %% pour changer l'espace entre les lignes












%%%%%%%%%%%%%%%%%%%%%%%%%%%%%%%%%%%%%%%%%%%%%%%%%%%%%%%%%%%%%%%%%%%%%%%%%%%%%%%
\title{------ Méthodes des Éléments Finis en 1D ------} %% choissez un titre approprié à votre sujet
\author{par\\MAAMIR, Mohamed\\ ZHANG, Xunjie\\ Compte-rendu du TP 1 de l'UE Méthodes des Éléments Finis} %% utilisez \\ pour une nouvelle ligne
\date{fait le 22 octobre 2016} %% pour afficher la date actuelle commenter cette ligne














%%%%%%%%%%%%%%%%%%%%%%%%%%%%%%%%%%%%%%%%%%%%%%%%%%%%%%%%%%%%%%%%%%%%%%%%%%%%%%%
%% TOUT ce qui va dans votre rapport doit être entre \begin{document} & \end{document}
\begin{document}
\selectlanguage{french} %% format français
\maketitle %% pour afficher le titre
\tableofcontents %% pour afficher/compiler le sommaire automatiquement
\listoffigures %% pour lister les figures













%%%%%%%%%%%%%%%%%%%%%%%%%%%%%%%%%%%%%%%%%%%%%%%%%%%%%%%%%%%%%%%%%%%%%%%%%%%%%%%
\chapter{Introduction et Modélisation par Éléments Finis}

\section{Introduction}

Dans le cadre de ce premier TP de MEF, nous avons étudié le déplacement des éléments d'une pale (de section variable)en rotation constante et soumise seulement à la force centrifuge.
Dans un premier temps, nous avons fait l'étude mathématique du modèle et trouvé une solution analytique des déplacements $u(r)$.Par la suite, grâce aux méthodes des éléments finis en 1D, nous avons construit un système linéaire à résoudre avec les outils numériques.
Nous avons programmé sur Python un algorithme permettant de construire les éléments du système à résoudre pour trouver les valeurs approchées de $u(r)$ sur différents points du maillage. La résolution de ce système de manière numérique nous a permis de construire cette approximation.
Par la suite, nous avons étudié les erreurs relatives et l'effet du changement de la densité du maillage dans la solution trouvée.
Ce compte-rendu décrit précisément les étapes de résolutions analytiques et numériques. De même, nous y analysons les résultats trouvés.

\section{Problème Physique}

\subsection{Équation d'équilibre}

Un petit élément de la pale de largeur $\delta r$ et de section $S(r)$ est soumis aux actions suivantes :
\begin{enumerate}
	\item Force centrifuge : $(\rho*S(r)*\omega^2*r)*\Delta r$
	\item Efforts internes : $ES(r+\Delta r)\frac{du}{dr}|_{r+\Delta r}-ES(r)\frac{du}{dr}|_{r}=\frac{d}{dr}*[ES(r)\frac{du}{dr}(r)]$
\end{enumerate}


On retrouve donc à l'état d'équilibre l'équation suivante :
\begin{equation}
	\frac{d}{dr}(ES(r)\frac{du}{dr})+(\rho S(r)\omega^2)r =0
	\label{eqequilibre}
\end{equation}

\section{Formulation Faible}

Pour une fonction test $v(r)$, on retrouve une formulation faible de la forme:
\begin{center}
Trouver $u(r)$ tel que $u(0)=0$ et $\frac{du}{dr}\Big|_{r=L}=0$ :
	\begin{equation}
		\int_0^L \!ES(r)\frac{du}{dr}~dr=\int_0^L \! \big(\rho S(r)\omega^2r\big)v(r)~dr
		\label{formulationfaible}
	\end{equation}
\end{center}


\subsection{solution Analytique}

L'équation d'équilibre est une \textbf{équation différentielle linéaire d'ordre deux}. Sa résolution est possible et est de manière relativement simple. De plus, nous avons \textbf{deux conditions limites} qui vont pouvoir nous aider à trouver les valeurs des constantes d'intégration.
\begin{enumerate}
	\item Dirichlet : $u(0)=0$
	\item Neumann :$\frac{du}{dr}\Big|_{r=L}=0$
\end{enumerate}

\subsubsection{Cas d'une Section Constante}

Si on considère la section $S_{const}$ comme étant constante, en intégrant l'équation d'équilibre (\ref{eqequilibre}) et en utilisant les conditions limites de Dirichlet et de Neumann, on retrouve une solution analytique de la forme:
\begin{equation}
	u(r)_{S_{const}}=\frac{\rho \omega^2}{2*E}\big(L^2 r-\frac{r^3}{3}\big)
	\label{solsconst}
\end{equation}

\subsubsection{Cas d'une Section non-constante}

Dans notre TP, nous avons considéré la section comme étant variable linéairement et étant de la forme :

$$S=S(r)=a*r+b$$
avec
$$a=\frac{S(L)-S(0)}{L},b=S(0)$$ 

En intégrant l'équation d'équilibre (\ref{eqequilibre})avec les mêmes conditions limites que précédemment on retrouve une solution analytique de la forme:
\begin{multline}
	u(r)_{S(r)}=\frac{\omega^2 \rho}{36a^3E}\Big( ar(-4a^2r^2-3abr+6b^2)-6(b-2aL)(aL+b)^2\ln(ar+b)\\+6\ln(b)(b-2aL)(aL+b)^2 \Big)
	\label{solution u}
\end{multline}

\textit{Résolu avec Wolfram Alpha\textregistered : \href{https://goo.gl/IB3dHC}{This link}}

\section{la forme d'approximation par interpolation}

\subsection{Interpolation d'ordre 1}

Dans le cas d'une interpolation d'ordre 1, chaque nœud d'un maillage est relié au suivant par une fonction affine de la forme : $f(x)=\alpha*x+\beta$.

Pour un maillage $M^h$ avec $N$ éléments $e_{k}$ et $N+1$ nœuds $r_{k}$, avec $k\in [0,N],k\in\mathbb{Z}$, nous pouvons facilement déduire la forme de l'interpolation sur chaque élément $e_{k}$ :

\begin{equation}
	u^h_{k}=\Big(\frac{U(r_{k})-U(r_{k-1})}{r_{k}-r_{k-1}}\Big)r+(\frac{r_{k}U(r_{k-1})-r_{k-1}U(r_{k})}{r_{k}-r_{k-1}}\Big)
	\label{interpolationP1}
\end{equation}




\subsection{Interpolation d'ordre 2}

\section{Expressions des matrices A et B }

(avec démonstrations)

%%%%%%%%%%%%%%%%%%%%%%%%%%%%%%%%%%%%%%%%%%%%%%%%%%%%%%%%%%%%%%%%%%%%%%%%%%%%%%%
\chapter{Résolution numérique} %% pour commencer un chapitre
\section{Algorithme}
\section{Programme}
- les commentaires sur la programmation/program, 
- validation du code.

%%%%%%%%%%%%%%%%%%%%%%%%%%%%%%%%%%%%%%%%%%%%%%%%%%%%%%%%%%%%%%%%%%%%%%%%%%%%%%%

\chapter{Résultats et Conclusions}

\section{Paramétrage du Problème}
avec des sections qui indiquent les résultats obtenus-demandés 
 des commentaires critiques, conclusion
finale qui synthétise tous les résultats principaux obtenus, et finalement,
perspectives de vos travaux.

\section{Résultats}
 

\subsection{Erreur relative en fontion de N}
peut importe N, l'erreur est tjr tres elevé sur le premier noeud,meme si les valeur sont tres nuls pour la solution analytique et la solution aprroché.
Cest du au approximation de l'ordinateur et du fait que les deux valeurs tendent vers zero. La solution analytique etant théoriquement zero, nous somme supposé trouvé une erreur relative infinie. On peux ne pas prendre en compte ce premier terme et se concentrer sur les suivant. On voit que l'érreur diminue 

on est en dessous de $err=0.06$ a partir de $h=$, soit $N \geq $
\section{Conclusion}

methode simple et rapide
requiere la connaissance de la solution analytique et du modèle pour connaitre a quel N la solution devient plus petites que lerreur tolerées

%\bibliography{MonBiblio.bib}
%\bibliographystyle{unsrtnat}	

\appendix 
\chapter{Code Python}

Les annexes (codes python/matlab/etc.)

\end{document}
\grid
